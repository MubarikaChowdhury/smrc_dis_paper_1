% Options for packages loaded elsewhere
\PassOptionsToPackage{unicode}{hyperref}
\PassOptionsToPackage{hyphens}{url}
\PassOptionsToPackage{dvipsnames,svgnames,x11names}{xcolor}
%
\documentclass[
]{interact}

\usepackage{amsmath,amssymb}
\usepackage{iftex}
\ifPDFTeX
  \usepackage[T1]{fontenc}
  \usepackage[utf8]{inputenc}
  \usepackage{textcomp} % provide euro and other symbols
\else % if luatex or xetex
  \usepackage{unicode-math}
  \defaultfontfeatures{Scale=MatchLowercase}
  \defaultfontfeatures[\rmfamily]{Ligatures=TeX,Scale=1}
\fi
\usepackage{lmodern}
\ifPDFTeX\else  
    % xetex/luatex font selection
\fi
% Use upquote if available, for straight quotes in verbatim environments
\IfFileExists{upquote.sty}{\usepackage{upquote}}{}
\IfFileExists{microtype.sty}{% use microtype if available
  \usepackage[]{microtype}
  \UseMicrotypeSet[protrusion]{basicmath} % disable protrusion for tt fonts
}{}
\makeatletter
\@ifundefined{KOMAClassName}{% if non-KOMA class
  \IfFileExists{parskip.sty}{%
    \usepackage{parskip}
  }{% else
    \setlength{\parindent}{0pt}
    \setlength{\parskip}{6pt plus 2pt minus 1pt}}
}{% if KOMA class
  \KOMAoptions{parskip=half}}
\makeatother
\usepackage{xcolor}
\setlength{\emergencystretch}{3em} % prevent overfull lines
\setcounter{secnumdepth}{5}
% Make \paragraph and \subparagraph free-standing
\ifx\paragraph\undefined\else
  \let\oldparagraph\paragraph
  \renewcommand{\paragraph}[1]{\oldparagraph{#1}\mbox{}}
\fi
\ifx\subparagraph\undefined\else
  \let\oldsubparagraph\subparagraph
  \renewcommand{\subparagraph}[1]{\oldsubparagraph{#1}\mbox{}}
\fi


\providecommand{\tightlist}{%
  \setlength{\itemsep}{0pt}\setlength{\parskip}{0pt}}\usepackage{longtable,booktabs,array}
\usepackage{calc} % for calculating minipage widths
% Correct order of tables after \paragraph or \subparagraph
\usepackage{etoolbox}
\makeatletter
\patchcmd\longtable{\par}{\if@noskipsec\mbox{}\fi\par}{}{}
\makeatother
% Allow footnotes in longtable head/foot
\IfFileExists{footnotehyper.sty}{\usepackage{footnotehyper}}{\usepackage{footnote}}
\makesavenoteenv{longtable}
\usepackage{graphicx}
\makeatletter
\def\maxwidth{\ifdim\Gin@nat@width>\linewidth\linewidth\else\Gin@nat@width\fi}
\def\maxheight{\ifdim\Gin@nat@height>\textheight\textheight\else\Gin@nat@height\fi}
\makeatother
% Scale images if necessary, so that they will not overflow the page
% margins by default, and it is still possible to overwrite the defaults
% using explicit options in \includegraphics[width, height, ...]{}
\setkeys{Gin}{width=\maxwidth,height=\maxheight,keepaspectratio}
% Set default figure placement to htbp
\makeatletter
\def\fps@figure{htbp}
\makeatother
% definitions for citeproc citations
\NewDocumentCommand\citeproctext{}{}
\NewDocumentCommand\citeproc{mm}{%
  \begingroup\def\citeproctext{#2}\cite{#1}\endgroup}
\makeatletter
 % allow citations to break across lines
 \let\@cite@ofmt\@firstofone
 % avoid brackets around text for \cite:
 \def\@biblabel#1{}
 \def\@cite#1#2{{#1\if@tempswa , #2\fi}}
\makeatother
\newlength{\cslhangindent}
\setlength{\cslhangindent}{1.5em}
\newlength{\csllabelwidth}
\setlength{\csllabelwidth}{3em}
\newenvironment{CSLReferences}[2] % #1 hanging-indent, #2 entry-spacing
 {\begin{list}{}{%
  \setlength{\itemindent}{0pt}
  \setlength{\leftmargin}{0pt}
  \setlength{\parsep}{0pt}
  % turn on hanging indent if param 1 is 1
  \ifodd #1
   \setlength{\leftmargin}{\cslhangindent}
   \setlength{\itemindent}{-1\cslhangindent}
  \fi
  % set entry spacing
  \setlength{\itemsep}{#2\baselineskip}}}
 {\end{list}}
\usepackage{calc}
\newcommand{\CSLBlock}[1]{\hfill\break\parbox[t]{\linewidth}{\strut\ignorespaces#1\strut}}
\newcommand{\CSLLeftMargin}[1]{\parbox[t]{\csllabelwidth}{\strut#1\strut}}
\newcommand{\CSLRightInline}[1]{\parbox[t]{\linewidth - \csllabelwidth}{\strut#1\strut}}
\newcommand{\CSLIndent}[1]{\hspace{\cslhangindent}#1}

\usepackage{orcidlink}
\makeatletter
\@ifpackageloaded{caption}{}{\usepackage{caption}}
\AtBeginDocument{%
\ifdefined\contentsname
  \renewcommand*\contentsname{Table of contents}
\else
  \newcommand\contentsname{Table of contents}
\fi
\ifdefined\listfigurename
  \renewcommand*\listfigurename{List of Figures}
\else
  \newcommand\listfigurename{List of Figures}
\fi
\ifdefined\listtablename
  \renewcommand*\listtablename{List of Tables}
\else
  \newcommand\listtablename{List of Tables}
\fi
\ifdefined\figurename
  \renewcommand*\figurename{Figure}
\else
  \newcommand\figurename{Figure}
\fi
\ifdefined\tablename
  \renewcommand*\tablename{Table}
\else
  \newcommand\tablename{Table}
\fi
}
\@ifpackageloaded{float}{}{\usepackage{float}}
\floatstyle{ruled}
\@ifundefined{c@chapter}{\newfloat{codelisting}{h}{lop}}{\newfloat{codelisting}{h}{lop}[chapter]}
\floatname{codelisting}{Listing}
\newcommand*\listoflistings{\listof{codelisting}{List of Listings}}
\makeatother
\makeatletter
\makeatother
\makeatletter
\@ifpackageloaded{caption}{}{\usepackage{caption}}
\@ifpackageloaded{subcaption}{}{\usepackage{subcaption}}
\makeatother
\ifLuaTeX
  \usepackage{selnolig}  % disable illegal ligatures
\fi
\usepackage{bookmark}

\IfFileExists{xurl.sty}{\usepackage{xurl}}{} % add URL line breaks if available
\urlstyle{same} % disable monospaced font for URLs
\hypersetup{
  pdftitle={Comparison of Confidence Interval in the OLS and Ridge Linear Regression Model: A Comparative study via Simulation},
  pdfauthor={Sultana Mubarika Rahman Chowdhury; B.M. Golam Kibria; Zoran Bursac},
  pdfkeywords={Multiple Linear Regression, Ridge
Regression, Multicolinearity},
  colorlinks=true,
  linkcolor={blue},
  filecolor={Maroon},
  citecolor={Blue},
  urlcolor={Blue},
  pdfcreator={LaTeX via pandoc}}

\title{Comparison of Confidence Interval in the OLS and Ridge Linear
Regression Model: A Comparative study via Simulation}
\author{Sultana Mubarika Rahman Chowdhury$\textsuperscript{1}$, B.M.
Golam Kibria$\textsuperscript{2}$, Zoran Bursac$\textsuperscript{1}$}

\thanks{CONTACT: Sultana Mubarika Rahman
Chowdhury. Email: \href{mailto:schow034@fiu.edu}{\nolinkurl{schow034@fiu.edu}}. }
\begin{document}
\captionsetup{labelsep=space}
\maketitle
\textsuperscript{1} Biosatistics Department, Florida International
University, Miami, Fl, USA\\ \textsuperscript{2} Department of
Mathematics and Statistics, Florida International University, Miami,
Fl, USA
\begin{abstract}
Write down the abstract here
\end{abstract}
\begin{keywords}
\def\sep{;\ }
Multiple Linear Regression\sep Ridge Regression\sep 
Multicolinearity
\end{keywords}

\section{Introduction}\label{introduction}

Multiple linear regression maps the relationship between two or more
predictors and dependent variable to a linear equation. The aim is to
predict the response variable using the independent variables. If X is a
n×p full rank matrix of predictors and Y is a n × 1 vector of response
variables the multiple linear regression can be explained as,

\[Y = X \times \beta + \epsilon, \] where, \(\beta\) is an p × 1 unknown
regression paramaters and \(\epsilon\) is the n × 1 vector of error with
mean zero and equal variance.Ordinary Least Square (OLS) method is
commonly used to estimate the unknown regression parameter. The OLS
estimate in case of linear regression model is defined as follows,

\[\hat{\beta} = (X^T X)^{-1} X^T y.\]

One of the key assumption of the widely used multiple linear regression
model is that the predictors need to be independent of each other.
Violating the assumption of independence results in an issue known as
multicollinearity. Multicollinearity, originally identified by Frisch
(1934), is the state in which two or more independent variables show a
strong correlation with one another. It causes standard error of the OLS
estimator to increase resulting in wide confidence intervals and less
reliable results. Identifying and interpreting significant predictors
also becomes difficult (Saleh, Arashi, and Kibria 2019). Increasing the
sample size, eliminating the highly correlated variables, Principal
component analysis are some of the ways to avoid multicollinearity but
with the risk of losing either valuable information or interpretability.

As the field of study progressed, researchers developed a number of
methods that outperforms OLS in presence of multicollinearity in data by
introducing bias to the estimator to gain smaller variance. Ridge
regression (Hoerl and Kennard 1970), Lasso (Tibshirani 1996), Stein
estimator (Stein et al. 1956), Modified ridge regression, Liu estimator
(Liu 1993), Kibria-Lukman estimator (Kibria and Lukman 2020) are some of
them. Among these, ridge regression has become one of the most popular
methods for dealing with multicollinearity, providing a robust
substitute for OLS. However, Ridge regression approach requires precise
ridge parameter estimation, because the number of parameters, sample
size, degree of correlation, and standard error vary greatly in
real-world data. As a results numerous ridge parameters estimators has
been suggested researchers till date. Mermi et al. (2024) in their
recent paper presented and compared a total of 366 different ridge
parameters estimators.

Typically, the performance of various ridge settings are compared to OLS
based on their mean square error (MSE). Nonetheless, certain unknown
characteristics in the model determines when Ridge Regression estimators
outperform Least Squares estimators in terms of MSE (Crivelli et al.
1995). However, that does not tell us how well the corresponding ridge
regression model performs in terms of finding out significance of the
predictors which is one of the key points in regression analysis.

There are two ways by which statistical significance of the independent
variables is determined. One is the method of hypothesis testing and the
other one is the method of constructing Confidence Interval (CI). On the
basis of power of the hypothesis test, a number of comparative studies
have been done for ridge regression with various tuning settings
Perez-Melo and Kibria (2020). Nevertheless, when evaluating regression
parameters, confidence intervals are preferable to hypothesis testing
because they offer a range of plausible values that represent the
accuracy, direction and magnitude of the estimate. They provide a more
precise grasp of practical significance based on sample size while
avoiding the drawbacks of p-values (Nickerson 2000) . In practical
fields such as, medical studies investigators are usually interested in
determining the size of difference of a measured outcome between groups,
rather than a simple indication of whether or not it is statistically
significant (Gardner and Altman 1986). Therefore, CI's can be considered
as more informative and transparent for making decisions.

This research examines the Confidence Intervals of multiple linear ridge
regression settings for a number of shrinkage parameter under identical
simulation conditions. The comparison is conducted based on the width of
the confidence interval and coverage probability. The research might be
useful to evaluate if shrinkage has produced more stable and dependable
estimates than ordinary least squares (OLS) regression by evaluating the
bias and variance reduction in Ridge Regression. We can also determine
which tuning parameter gives a high coverage probability with a
comparatively narrow indicating higher precision.

The rest of the paper is organized as follows,

\section{Statistical Methodology}\label{statistical-methodology}

\subsection{Ridge Regression
Estimators}\label{ridge-regression-estimators}

\subsection{Confidence Interval}\label{confidence-interval}

\subsection{CI for Ridge regression}\label{ci-for-ridge-regression}

\section{Simulation Study}\label{simulation-study}

\section{Application}\label{application}

\section{Discussion and Conclusion}\label{discussion-and-conclusion}

\section*{Bibliography}\label{bibliography}
\addcontentsline{toc}{section}{Bibliography}

\phantomsection\label{refs}
\begin{CSLReferences}{1}{0}
\bibitem[\citeproctext]{ref-crivelli1995confidence}
Crivelli, Ana, Luis Firinguetti, Rosa Montano, and Margarita Munóz.
1995. {``Confidence Intervals in Ridge Regression by Bootstrapping the
Dependent Variable: A Simulation Study.''} \emph{Communications in
Statistics-Simulation and Computation} 24 (3): 631--52.

\bibitem[\citeproctext]{ref-cule2011significance}
Cule, Erika, Paolo Vineis, and Maria De Iorio. 2011. {``Significance
Testing in Ridge Regression for Genetic Data.''} \emph{BMC
Bioinformatics} 12: 1--15.

\bibitem[\citeproctext]{ref-frisch1934statistical}
Frisch, Ragnar. 1934. {``Statistical Confluence Analysis by Means of
Complete Regression Systems.''} \emph{(No Title)}.

\bibitem[\citeproctext]{ref-gardner1986confidence}
Gardner, Martin J, and Douglas G Altman. 1986. {``Confidence Intervals
Rather Than p Values: Estimation Rather Than Hypothesis Testing.''}
\emph{Br Med J (Clin Res Ed)} 292 (6522): 746--50.

\bibitem[\citeproctext]{ref-gokpinar2016study}
Gökpınar, Esra, and Meral Ebegil. 2016. {``A Study on Tests of
Hypothesis Based on Ridge Estimator.''} \emph{Gazi University Journal of
Science} 29 (4): 769--81.

\bibitem[\citeproctext]{ref-halawa2000tests}
Halawa, AM, and MY El Bassiouni. 2000. {``Tests of Regression
Coefficients Under Ridge Regression Models.''} \emph{Journal of
Statistical Computation and Simulation} 65 (1-4): 341--56.

\bibitem[\citeproctext]{ref-hoerl1970ridge}
Hoerl, Arthur E, and Robert W Kennard. 1970. {``Ridge Regression: Biased
Estimation for Nonorthogonal Problems.''} \emph{Technometrics} 12 (1):
55--67.

\bibitem[\citeproctext]{ref-kibria2020new}
Kibria, BM Golam, and Adewale F Lukman. 2020. {``A New Ridge-Type
Estimator for the Linear Regression Model: Simulations and
Applications.''} \emph{Scientifica} 2020 (1): 9758378.

\bibitem[\citeproctext]{ref-liu1993new}
Liu, Kejian. 1993. {``A New Class of Biased Estimate in Linear
Regression.''} \emph{Communications in Statistics - Theory and Methods}
22 (2): 393--402. \url{https://doi.org/10.1080/03610929308831027}.

\bibitem[\citeproctext]{ref-mermi2024new}
Mermi, Selman, Özge Akkuş, Atila Göktaş, and Necla Gündüz. 2024. {``A
New Robust Ridge Parameter Estimator Having No Outlier and Ensuring
Normality for Linear Regression Model.''} \emph{Journal of Radiation
Research and Applied Sciences} 17 (1): 100788.

\bibitem[\citeproctext]{ref-nickerson2000null}
Nickerson, Raymond S. 2000. {``Null Hypothesis Significance Testing: A
Review of an Old and Continuing Controversy.''} \emph{Psychological
Methods} 5 (2): 241.

\bibitem[\citeproctext]{ref-perez2020some}
Perez-Melo, Sergio, and BM Golam Kibria. 2020. {``On Some Test
Statistics for Testing the Regression Coefficients in Presence of
Multicollinearity: A Simulation Study.''} \emph{Stats} 3 (1): 40--55.

\bibitem[\citeproctext]{ref-saleh2019theory}
Saleh, AK Md Ehsanes, Mohammad Arashi, and BM Golam Kibria. 2019.
\emph{Theory of Ridge Regression Estimation with Applications}. John
Wiley \& Sons.

\bibitem[\citeproctext]{ref-stein1956inadmissibility}
Stein, Charles et al. 1956. {``Inadmissibility of the Usual Estimator
for the Mean of a Multivariate Normal Distribution.''} In
\emph{Proceedings of the Third Berkeley Symposium on Mathematical
Statistics and Probability}, 1:197--206. 1.

\bibitem[\citeproctext]{ref-tibshirani1996regression}
Tibshirani, Robert. 1996. {``Regression Shrinkage and Selection via the
Lasso.''} \emph{Journal of the Royal Statistical Society Series B:
Statistical Methodology} 58 (1): 267--88.

\end{CSLReferences}



\end{document}
